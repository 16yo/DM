\documentclass{article}

\usepackage{amsmath}
\usepackage[russian]{babel}
\usepackage[margin=2.5cm]{geometry}
\usepackage{amssymb}

\setcounter{MaxMatrixCols}{20}

\title{РГР по дискретной математике\\Четвертая задача}
\author{Ахметшин Б. Р. -- М8О-103Б-22 -- 2 вариант}
\date{Май, 2023}

\begin{document}
\maketitle

\section*{Дано}
\begin{enumerate}
    \item[а)] 1 0 0 1
    \item[б)] 1 0 1 1 1 0 1
    \item[в)] 0 1 1 1 0 0 0
\end{enumerate}


\section*{Задание}
Рассматривается (4, 7) –- код Хэмминга. Для слова а) определить соответствующее ему
кодовое слово. Пусть при приеме каждого из слов б), в) возможно была допущена ошибка
(не более чем в одной позиции). Определить наличие и положение ошибки. Какие слова
были переданы? Какие слова были закодированы?


\section*{Решение}
По условию $(m, n) = (4, 7) \Rightarrow r = 3$
\\
$b = b_1\dots b_7$, где $b_1, b_2, b_4$ - вспомогательные символы, остальные - символы сообщения
\\
Вспомогательная матрица
$
M =
\begin{pmatrix}
    0 && 0 && 0 && 1 && 1 && 1 && 1 \\
    0 && 1 && 1 && 0 && 0 && 1 && 1 \\
    1 && 0 && 1 && 0 && 1 && 0 && 1
\end{pmatrix}
$
\\
$b \cdot M^T = 0 \Rightarrow
\begin{cases}
    b_4 + b_5 + b_6 + b_7 = 0 \\
    b_2 + b_3 + b_6 + b_7 = 0 \\
    b_1 + b_3 + b_5 + b_7 = 0
\end{cases}
$

\subsection*{Пункт a}
$a = 1001 = b_3b_5b_6b_7 \Rightarrow
\begin{cases}
    b_4 + 0 + 0 + 1 = 0 \\
    b_2 + 1 + 0 + 1 = 0 \\
    b_1 + 1 + 0 + 1 = 0
\end{cases} \Rightarrow
b_4 = 1, b_2 = 0, b_1 = 0 \Rightarrow
\mathbf{b = 0011001}
$

\subsection*{Пункт б}
$a = 1011101$\\\\
$a \cdot M^T \Rightarrow
\begin{cases}
    1 + 1 + 0 + 1 = 1 \\
    0 + 1 + 0 + 1 = 0 \\
    1 + 1 + 1 + 1 = 0
\end{cases}
$
-- не все значения равны нулю, следовательно есть ошибка:\\
$100_2 = 4_{10}$, следовательно ошибка в четвертой позиции слова, тогда
изначальное переданное слово было равно \textbf{1010101}, а закодированное
слово было \textbf{1101}
\\
Закодированное слово было \textbf{1101}

\subsection*{Пункт в}
$a = 0111000$
\\\\
$a \cdot M^T \Rightarrow
\begin{cases}
    1 + 0 + 0 + 0 = 1 \\
    1 + 1 + 0 + 0 = 0 \\
    0 + 1 + 0 + 0 = 1
\end{cases}
$
-- не все значения равны нулю, следовательно есть ошибка:\\
$101_2 = 5_{10}$, следовательно ошибка в пятой позиции слова, тогда
изначальное переданное слово было равно \textbf{0111100}, а закодированное
слово было \textbf{1100}
\section*{Ответ}
\begin{enumerate}
    \item \textbf{b = 0011001}
    \item \textbf{1101}
    \item \textbf{1100} 
\end{enumerate}


\end{document}
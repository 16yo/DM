\documentclass{article}

\usepackage{amsmath}
\usepackage{amssymb}
\usepackage[russian]{babel}


\title{Расчетно-графическая работа по дискретной математике\\Первая задача}
\author{Ахметшин Б.Р. -- М8О-103Б-22 -- 2 вариант}
\date{Март, 2023}

\begin{document}

\maketitle

\section*{Дано}
Заданы функции на множестве целых неотрицательных чисел $\mathbb{N}_0$:
\[S(x) = x + 1, O(x) = 0, I^n_m(x_1, ..., x_n) = x_m (1 <= m <= n), \sigma(x_1, x_2) = x_1 + x_2.\]
\section*{Найти}
\[F(x,y) = 3x(y+2)\]

\section*{Решение}
Пусть известно выражение функции \[M(x,y) = xy\]
Тогда искомая функция может быть выражена как \[F(x,y) = M(3,M(x,\sigma(y,2)))\]
Выразим функцию $M(x,y)$:
\[M(x,y) = xy = \underbrace{x + ... + x}_{y}=\underbrace{\sigma(x,\sigma(x,...)}_{y}\]
Т.е. ф-я $M(x,y)$ может быть задана при помощи оператора примитивной рекурсии:
\[
M(x,y) =
\begin{cases}
  M(x,0)=O(x)\\
  M(x,y+1)=\sigma(x, M(x,y))
\end{cases}
\]
\\\\
Ответ: $F(x,y)=M(3,M(x,\sigma(y,2)))$ при заданной функции $M(x,y)$.
\end{document}
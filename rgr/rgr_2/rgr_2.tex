\documentclass{article}

\usepackage{amsmath}
\usepackage[russian]{babel}
\usepackage[margin=2.5cm]{geometry}


\title{РГР по дискретной математике\\Вторая задача}
\author{Клименко В. М. -- М8О-103Б-22 -- 2 вариант}
\date{Апрель, 2023}

\begin{document}
\maketitle


\section*{Дано}
$S_8 \ni \pi = [(85214)(6231)(8145)(4726)]^{-85}$ 

\section*{Задание}
Для заданной подстановки из $S_8$ определить:
\begin{enumerate}
    \item разложение в произведение независимых циклов
    \item порядок подстановки
    \item разложение в произведение транспозиций
    \item четность подстановки
\end{enumerate}


\section*{Решение}
\subsection*{Пункт 1}
\begin{enumerate}
\item $\pi = [(85214)(6231)(8145)(4726)]^{-85} = [(85214)(6231)(8147265)]^{-85} = \\
= [(85214)(1473)(586)]^{-85} = [(1852)(347)(586)]^{-85} 
= [(347)(1862)]^{-85} $.

\item Пусть $\pi = (\alpha \cdot \beta)^{-85}$, где $\alpha = (1862), \beta = (347)$ \\
$\alpha^{4} = e, \beta^{3} = e \Rightarrow (\alpha \cdot \beta)^{12} = e$

\item $\pi = (\alpha \cdot \beta)^{-85} = (\alpha \cdot \beta)^{12 * (-7) + (-1)} = (\alpha \cdot \beta)^{-1} = (1268)(374)$

\end{enumerate}

\subsection*{Пункт 2}
Количество элементов $\pi \in S_8$ равно 8, следовательно порядок подстановки тоже $8$.

\subsection*{Пункт 3}
$\pi = (1268)(374) = (18)(16)(12)(34)(37)$.

\subsection*{Пункт 4}
Число транспозиций -- $5$, следовательно подстановка нечетная.


\section*{Ответ}
\begin{enumerate}
    \item $(1268)(374)$
    \item $8$
    \item $(18)(16)(12)(34)(37)$
    \item нечетная
\end{enumerate}

\end{document}
\documentclass{article}

\usepackage{amsmath}
\usepackage[russian]{babel}
\usepackage[margin=2.5cm]{geometry}
\usepackage{amssymb}


\title{РГР по дискретной математике\\Пятая задача}
\author{Ахметшин Б. Р. -- М8О-103Б-22 -- 2 вариант}
\date{Май, 2023}

\begin{document}
\maketitle


\section*{Дано}
Целые числа <Z, +, $\times$>


\section*{Задание}
Определить, является ли полем или кольцом заданная алгебраическая структура.
Проверить, существуют ли делители нуля.

\section*{Решение}
\subsection*{Сложение}
\begin{enumerate}
    \item Коммутативность, ассоциативность, замкнутость - по аксиомам сложения
    \item Единичный елемент: $e$ = 0
    \item Обратный элемент: $\forall a \in Z$  $\exists \overline{a}: a + \overline{a} = \overline{a} + a = e$ 
\end{enumerate}
$\Rightarrow$ $<Z, +>$ - коммутативная группа

\subsection*{Умножение}
\begin{enumerate}
    \item Коммутативность, ассоциативность, замкнутость - по аксиомам умножения
    \item Единичный элемент $e = 1$
    \item Обратный элемент существует только для единицы: $a \in Z$  $\exists \overline{a} \Rightarrow a = e$
\end{enumerate}
$\Rightarrow$ $<Z, \times>$ - коммутативный моноид

\subsection*{Дистрибутивность}

Дистрибутивность умножения по сложению: $\forall a, b, c \in Z: a \times (b + c) = a \times b + a \times c$.\\
Дистрибутивности сложения по умножению нет.

\subsection*{Делители нуля}

Делителей нуля нет.

\section*{Ответ}

$<Z, +, \times>$ - коммутативное кольцо.

\end{document}

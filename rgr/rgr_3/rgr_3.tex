\documentclass{article}

\usepackage{amsmath}
\usepackage[russian]{babel}
\usepackage[margin=2.5cm]{geometry}
\usepackage{booktabs}

\newcommand\headercell[1]{\smash[b]{\begin{tabular}[t]{@{}c@{}} #1 \end{tabular}}}


\title{РГР по дискретной математике\\Третья задача}
\author{Ахметшин. Б.Р. -- М8О-103Б-22 -- 2 вариант}
\date{Апрель, 2023}

\begin{document}
\maketitle


\section*{Дано}
$H = <(1243), (23)>$


\section*{Задание}
Определить для заданной подгруппы $H \subset S_4$:
\begin{enumerate}
    \item элементы из $H$
    \item левые смежные классы группы $S_4$ по $H$
    \item правые смежные классы группы $S_4$ по $H$
    \item является ли $H$ нормальной подгруппой
\end{enumerate}


\section*{Решение}
% $S_4 = \{
% \pi_0,
% (12), (13), (14), (23), (24), (34),
% (123), (124), (132), (134), (142), (143), (234), (243),\\
% (1234), (1243), (1324), (1342), (1423), (1432)
% \}$

\subsection*{Пункт 1}
$(1243)(23) = (12)(34)$, $(23)(1243) = (13)(24)\\\\$
\begin{tabular}{l|llllllll}
      $\cdot$    & $\pi_0$    & $(1243)$   & $(23)$     & $(12)(34)$ & $(13)(24)$ & $(14)$     & $(1342)$   & $(14)(23)$  \\ 
    \midrule
      $\pi_0$    & $\pi_0$    & $(1243)$   & $(23)$     & $(12)(34)$ & $(13)(24)$ & $(14)$     & $(1342)$   & $(14)(23)$  \\
      $(1243)$   & $(1243)$   & $(14)(23)$ & $(12)(34)$ & $(14)$     & $(23)$     & $(13)(24)$ & $\pi_0$    & $(1342)$    \\
      $(23)$     & $(23)$     & $(13)(24)$ & $\pi_0$    & $(1324)$   & $(1243)$   & $(14)(23)$ & $(12)(34)$ & $(14)$      \\
      $(12)(34)$ & $(12)(34)$ & $(23)$     & $(1243)$   & $\pi_0$    & $(14)(23)$ & $(1342)$   & $(14)$     & $(13)(24)$  \\
      $(13)(24)$ & $(13)(24)$ & $(14)$     & $(1324)$   & $(14)(23)$ & $\pi_0$    & $(1243)$   & $(23)$     & $(12)(34)$  \\
      $(14)$     & $(14)$     & $(12)(34)$ & $(14)(23)$ & $(1243)$   & $(1342)$   & $\pi_0$    & $(13)(24)$ & $(23)$      \\
      $(1342)$   & $(1342)$   & $\pi_0$    & $(13)(24)$ & $(23)$     & $(14)$     & $(12)(34)$ & $(14)(23)$ & $(1243)$    \\
      $(14)(23)$ & $(14)(23)$ & $(1342)$   & $(14)$     & $(13)(24)$ & $(12)(34)$ & $(23)$     & $(1243)$   & $\pi_0$     \\
\end{tabular}
\\\\
$H = \{\pi_0, (14), (23), (12)(34), (13)(24), (14)(23), (1243), (1342)\}$

\subsection*{Пункт 2}
Классы эквиваленции левых смежных классов $H$
\begin{enumerate}
    \item $\pi_0 \cdot  H = \{\pi_0, (14), (23), (12)(34), (13)(24), (14)(23), (1243), (1342)\}$
    \item $(34) \cdot  H = \{(34), (134), (243), (12), (1423), (1324), (123), (142)\}$
    \item $(234) \cdot  H = \{(234), (1234), (24), (132), (143), (124), (13), (1432)\}$
\end{enumerate}

\subsection*{Пункт 3}
Классы эквиваленции правых смежных классов $H$
\begin{enumerate}
    \item $H \cdot \pi_0  = \{\pi_0, (14), (23), (12)(34), (13)(24), (14)(23), (1243), (1342)\}$
    \item $H \cdot (34) = \{(34), (134), (243), (12), (1423), (1324), (123), (142)\}$
    \item $H \cdot (234) = \{(234), (1234), (24), (132), (143), (124), (13), (1432)\}$
\end{enumerate}

\subsection*{Пункт 4}
$H$ является нормальной подгруппой, так как ЛСК $=$ ПСК


\end{document}
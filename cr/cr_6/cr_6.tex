\documentclass{article}

\usepackage{amsmath, systeme}
\usepackage[russian]{babel}
\usepackage[margin=2.5cm]{geometry}
\usepackage{booktabs}

\usepackage{tikz}
\usetikzlibrary{graphs, babel, quotes, calc, arrows.meta}

\title{Курсовая работа по дискретной математике\\Шестая задача}
\author{Ахметшин Б. Р. -- М8О-103Б-22 -- 2 вариант}
\date{Май, 2023}


\begin{document}

\maketitle

\section*{Задача}
\tolerance 1000
\large
Пусть каждому ребру неориентированного графа соответствует некоторый элемент
электрической цепи. Составить линейно независимые системы уравнений Кирхгофа для токов и
напряжений. Пусть первому и пятому ребру соответствуют источники тока с ЭДС $E_1$ и $E_2$
(полярность выбирается произвольно), а остальные элементы являются сопротивлениями. Используя
закон Ома, и, предполагая внутренние сопротивления источников тока равными нулю, получить
систему уравнений для токов.

\section*{Дано}

\begin{center}
    \begin{tikzpicture}[circ/.style={circle, draw}]
        \path[nodes={circ}]
            (0, 3) node(1) {1}
            (4, 3) node(2) {2}
            (8, 3) node(3) {3}
            (0, 0) node(4) {4}
            (6, 0) node(5) {5}
        ;
        \draw[nodes={auto}]
            (1) -- node{$q_2$} (2)
            (1) -- node{$q_1$} (4)
            (1) -- node{$q_6$} (5)
            (2) -- node{$q_7$} (3)
            (2) -- node{$q_5$} (4)
            (2) -- node{$q_3$} (5)
            (3) -- node{$q_8$} (5)
            (4) -- node{$q_4$} (5)
        ;
    \end{tikzpicture}
\end{center}

$U = 
\begin{pmatrix}
    E_1 \\
    U_2 \\
    U_3 \\
    U_4 \\
    E_5 \\
    U_6 \\
    U_7 \\
    U_8 
\end{pmatrix}
$

\section*{Решение}

\begin{enumerate}
    \item Зададим ориентацию
        \begin{center}
            \begin{tikzpicture}[circ/.style={circle, draw}]
                \path[nodes={circ}]
                    (0, 3) node(1) {1}
                    (4, 3) node(2) {2}
                    (8, 3) node(3) {3}
                    (0, 0) node(4) {4}
                    (6, 0) node(5) {5}
                ;
                \draw[-{Stealth[scale=2]}, nodes={auto}]
                    (1) -- node{$q_2$} (2);
                \draw[-{Stealth[scale=2]}, nodes={auto}]
                    (2) -- node{$q_7$} (3);
                \draw[-{Stealth[scale=2]}, nodes={auto}]
                    (4) -- node{$q_5$} (2);
                \draw[-{Stealth[scale=2]}, nodes={auto}]
                    (4) -- node{$q_1$} (1);
                \draw[-{Stealth[scale=2]}, nodes={auto}]
                    (5) -- node{$q_4$} (4);
                \draw[-{Stealth[scale=2]}, nodes={auto}]
                    (5) -- node{$q_6$} (1); 
                \draw[-{Stealth[scale=2]}, nodes={auto}]
                    (5) -- node{$q_3$} (2); 
                \draw[-{Stealth[scale=2]}, nodes={auto}]
                    (5) -- node{$q_8$} (3); 
            \end{tikzpicture}
        \end{center}
    \item Построим остовное дерево $D$
    \begin{center}
        \begin{tikzpicture}[circ/.style={circle, draw}]
            \path[nodes={circ}]
                (0, 3) node(1) {1}
                (4, 3) node(2) {2}
                (8, 3) node(3) {3}
                (0, 0) node(4) {4}
                (6, 0) node(5) {5}
            ;
            \draw[-{Stealth[scale=2]}, nodes={auto}]
                (1) -- node{$q_2$} (2);
            \draw[-{Stealth[scale=2]}, nodes={auto}]
                (2) -- node{$q_7$} (3);
            \draw[-{Stealth[scale=2]}, nodes={auto}]
                (4) -- node{$q_1$} (1);
            \draw[-{Stealth[scale=2]}, nodes={auto}]
                (5) -- node{$q_4$} (4);
            ;
        \end{tikzpicture}
    \end{center}
    \item Добавляем по одному ребру из графа, получаем ровно один простой цикл.
    Записываем соответствующие вектор циклы. 
    \begin{flushleft}
        $(D + q_5)$ $\mu_1$: $v_4-v_1-v_2-v_4 \rightarrow C(\mu_1) = (1, 1, 0, 0, -1, 0, 0, 0)$\\
        $(D + q_6)$ $\mu_2$: $v_5-v_4-v_1-v_5 \rightarrow C(\mu_2) = (1, 0, 0, 1, 0, -1, 0, 0)$\\
        $(D + q_3)$ $\mu_3$: $v_5-v_4-v_1-v_2-v_5 \rightarrow C(\mu_3) = (1, 1, -1, 1, 0, 0, 0, 0)$\\
        $(D + q_8)$ $\mu_4$: $v_5-v_4-v_1-v_2-v_3-v_5 \rightarrow C(\mu_4) = (1, 1, 0, 1, 0, 0, 1, -1)$\\
    \end{flushleft}
    $\Rightarrow C = 
    \begin{pmatrix}
        1 & 1 & 0 & 0 & -1 & 0 & 0 & 0 \\
        1 & 0 & 0 & 1 & 0 & -1 & 0 & 0 \\
        1 & 1 & -1 & 1 & 0 & 0 & 0 & 0 \\
        1 & 1 & 0 & 1 & 0 & 0 & 1 & -1
    \end{pmatrix}
    $
    \item По закону Кирхгофа для напряжений $C \times U = 0$
    \newline
    $
    \begin{pmatrix}
        1 & 1 & 0 & 0 & -1 & 0 & 0 & 0 \\
        1 & 0 & 0 & 1 & 0 & -1 & 0 & 0 \\
        1 & 1 & -1 & 1 & 0 & 0 & 0 & 0 \\
        1 & 1 & 0 & 1 & 0 & 0 & 1 & -1
    \end{pmatrix}
    \times
    \begin{pmatrix}
        E_1 \\
        U_2 \\
        U_3 \\
        U_4 \\
        E_5 \\
        U_6 \\
        U_7 \\
        U_8 
    \end{pmatrix}
    =
    \begin{pmatrix}
        E_1 + U_2 - E_5 \\
        E_1 + U_4 - U_6 \\
        E_1 + U_2 - U_3 + U_4 \\
        E_1 + U_2 + U_4 + U_7 - U_8 
    \end{pmatrix}
    =
    \begin{pmatrix}
        0 \\
        0 \\
        0 \\
        0
    \end{pmatrix}
    $
    \newline
    \begin{equation*}
    \Rightarrow
        \begin{cases}
            E_1 + U_2 - E_5 = 0 \\
            E_1 + U_4 - U_6 = 0 \\
            E_1 + U_2 - U_3 + U_4 = 0 \\
            E_1 + U_2 + U_4 + U_7 - U_8 = 0        
        \end{cases}
    \Leftrightarrow
        \begin{cases}
            U_2 = E_5 - E_1 \\
            U_4 = U_6 - E_1 \\
            U_3 = E_5 + U_4 \\
            U_7 = U_8 - U_6 + E_1 - E_5
        \end{cases}
    \end{equation*}
    \item Находим матрицу инцидентности $B$\\
    \newline
    B = 
    $
    \begin{pmatrix}
        1  & -1 & 0  & 1  & 0  & 0  & 0  & 0 \\
        0  & 1  & 1  & 0  & 1  & 0  & -1 & 0 \\
        0  & 0  & 0  & 0  & 0  & 0  & 1  & 1 \\
        -1 & 0  & 0  & 1  & -1 & 0  & 0  & 0 \\
        0  & 0  & 1  & 1  & 0  & 1  & 0  & 1
    \end{pmatrix}
    $
    \item По закону Кирхгофа для токов $B \times I = 0$
    \newline
    $
    \begin{pmatrix}
        1  & -1 & 0  & 0  & 1  & 0  & 0  & 0 \\
        0  & 1  & 1  & 0  & 1  & 0  & -1 & 0 \\
        0  & 0  & 0  & 0  & 0  & 0  & 1  & 1 \\
        -1 & 0  & 0  & 1  & -1 & 0  & 0  & 0 \\
        0  & 0  & 1  & 1  & 0  & 1  & 0  & 1
    \end{pmatrix}
    \times
    \begin{pmatrix}
        I_1 \\
        I_2 \\
        I_3 \\
        I_4 \\
        I_5 \\
        I_6 \\
        I_7 \\
        I_8
    \end{pmatrix}
    =
    \begin{pmatrix}
        I_1 - I_2 + I_5       \\
        I_2 + I_3 + I_5 - I_7 \\
        I_7 + I_8             \\
        -I_1 + I_4 - I_5      \\
        I_3 + I_4 + I_6 + I_8
    \end{pmatrix}
    =
    \begin{pmatrix}
        0 \\
        0 \\
        0 \\
        0 \\
        0
    \end{pmatrix}
    $
    \newline
    \begin{equation*}
        \Rightarrow
        \begin{cases}
            I_1 - I_2 + I_5 = 0       \\
            I_2 + I_3 + I_5 - I_7 = 0 \\
            I_7 + I_8 = 0             \\
            -I_1 + I_4 - I_5 = 0
        \end{cases},
    \end{equation*}
    с учетом того, что $rgB = 4$.
    \item Вместе с законом Ома получается система:
    \begin{equation*}
        \begin{cases}
            I_2 R_2 = E_5 - E_1             \\
            I_4 R_4 = I_6 R_6 - E_1             \\
            I_3 R_3 = E_5 + I_4            \\
            I_7 R_7 = I_8 R_8 - I_6 R_6 + E_1 - E_5 \\
            I_1 - I_2 + I_5 = 0       \\
            I_2 + I_3 + I_5 - I_7 = 0 \\
            I_7 + I_8 = 0             \\
            -I_1 + I_4 - I_5 = 0
        \end{cases}
    \end{equation*}
\end{enumerate}



\end{document}
\documentclass{article}

\usepackage{amsmath, systeme}
\usepackage[russian]{babel}
\usepackage[margin=2.5cm]{geometry}
\usepackage{booktabs}

\usepackage{tikz}
\usetikzlibrary{graphs, babel, quotes, calc, arrows.meta}


\title{Курсовая работа по дискретной математике\\Седьмая задача}
\author{Ахметшин Б. Р. -- М8О-103Б-22 -- 2 вариант}
\date{Май, 2023}

\begin{document}

\maketitle

\section*{Задача}
Построить максимальный поток по данной транспортной сети.

\section*{Дано}

\begin{center}
    \begin{tikzpicture}[circ/.style={circle, draw}]
        \path[nodes={circ}]
            (6,  0) node(6) {6}
            (2,  1) node(3) {3}
            (10, 1) node(8) {8}
            (0,  3) node(1) {1}
            (6,  3) node(5) {5}
            (12, 3) node(9) {9}
            (2,  5) node(2) {2}
            (6,  6) node(4) {4}
            (10, 5) node(7) {7}
        ;
        \draw[-{Stealth[scale=2]}, nodes={auto}]
            (1) -- node{4} (2);
        \draw[-{Stealth[scale=2]}, nodes={auto}]
            (1) -- node{3} (3);
        \draw[-{Stealth[scale=2]}, nodes={auto}]
            (1) -- node{8} (4);
        \draw[-{Stealth[scale=2]}, nodes={auto}]
            (1) -- node{7} (5);
        \draw[-{Stealth[scale=2]}, nodes={auto}]
            (1) -- node{10} (6);
        \draw[-{Stealth[scale=2]}, nodes={auto}]
            (2) -- node{4} (4);
        \draw[-{Stealth[scale=2]}, nodes={auto}]
            (2) -- node{3} (5);
        \draw[-{Stealth[scale=2]}, nodes={auto}]
            (3) -- node{5} (5);
        \draw[-{Stealth[scale=2]}, nodes={auto}]
            (3) -- node{3} (6);
        \draw[-{Stealth[scale=2]}, nodes={auto}]
            (4) -- node{7} (7);
        \draw[-{Stealth[scale=2]}, nodes={auto}]
            (5) -- node{4} (7);
        \draw[-{Stealth[scale=2]}, nodes={auto}]
            (5) -- node{4} (8);
        \draw[-{Stealth[scale=2]}, nodes={auto}]
            (5) -- node{6} (9);
        \draw[-{Stealth[scale=2]}, nodes={auto}]
            (6) -- node{10} (8);
        \draw[-{Stealth[scale=2]}, nodes={auto}]
            (7) -- node{10} (9);
        \draw[-{Stealth[scale=2]}, nodes={auto}]
            (8) -- node{15} (9);
    \end{tikzpicture}
\end{center}

\newpage
\section*{Решение}

\begin{enumerate}
    \item Построим полный поток
    \begin{center}
        \begin{tikzpicture}[circ/.style={circle, draw}]
            \path[nodes={circ}]
                (6,  0) node(6) {6}
                (2,  1) node(3) {3}
                (10, 1) node(8) {8}
                (0,  3) node(1) {1}
                (6,  3) node(5) {5}
                (12, 3) node(9) {9}
                (2,  5) node(2) {2}
                (6,  6) node(4) {4}
                (10, 5) node(7) {7}
            ;
            \draw[-{Stealth[scale=2]}, nodes={auto}]
                (1) -- node{0+4} (2);
            \draw[-{Stealth[scale=2]}, nodes={auto}]
                (1) -- node{0+3} (3);
            \draw[-{Stealth[scale=2]}, nodes={auto}]
                (1) -- node{0+3} (4);
            \draw[-{Stealth[scale=2]}, nodes={auto}]
                (1) -- node{0+1+6} (5);
            \draw[-{Stealth[scale=2]}, nodes={auto}]
                (1) -- node{0+7} (6);
            \draw[-{Stealth[scale=2]}, nodes={auto}]
                (2) -- node{0+4} (4);
            \draw[-{Stealth[scale=2]}, nodes={auto}]
                (2) -- node{0} (5);
            \draw[-{Stealth[scale=2]}, nodes={auto}]
                (3) -- node{0} (5);
            \draw[-{Stealth[scale=2]}, nodes={auto}]
                (3) -- node{0+3} (6);
            \draw[-{Stealth[scale=2]}, nodes={auto}]
                (4) -- node{0+4+3} (7);
            \draw[-{Stealth[scale=2]}, nodes={auto}]
                (5) -- node{0+1} (7);
            \draw[-{Stealth[scale=2]}, nodes={auto}]
                (5) -- node{0} (8);
            \draw[-{Stealth[scale=2]}, nodes={auto}]
                (5) -- node{0+6} (9);
            \draw[-{Stealth[scale=2]}, nodes={auto}]
                (6) -- node{0+3+7} (8);
            \draw[-{Stealth[scale=2]}, nodes={auto}]
                (7) -- node{0+4+3+1} (9);
            \draw[-{Stealth[scale=2]}, nodes={auto}]
                (8) -- node{0+3+7} (9);
        \end{tikzpicture}
    \end{center}

    \begin{enumerate}
        \item $v_1-v_2-v_4-v_7-v_9$ \\
        $min\{4, 4, 7, 10\} = 4$
        \item $v_1-v_4-v_7-v_9$ \\
        $min\{8, 7-4, 10-4\} = 3$
        \item $v_1-v_5-v_7-v_9$ \\
        $min\{7, 4, 10-4-3\} = 1$
        \item $v_1-v_5-v_9$ \\
        $min\{7-1, 6\} = 6$
        \item $v_1-v_3-v_6-v_8-v_9$ \\
        $min\{3, 3, 10, 15\} = 3$
        \item $v_1-v_6-v_8-v_9$ \\
        $min\{10, 10-3, 15-3\} = 7$
        
    \end{enumerate}

    Получился полный поток:
    \begin{center}
        \begin{tikzpicture}[circ/.style={circle, draw}]
            \path[nodes={circ}]
                (6,  0) node(6) {6}
                (2,  1) node(3) {3}
                (10, 1) node(8) {8}
                (0,  3) node(1) {1}
                (6,  3) node(5) {5}
                (12, 3) node(9) {9}
                (2,  5) node(2) {2}
                (6,  6) node(4) {4}
                (10, 5) node(7) {7}
            ;
            \draw[-{Stealth[scale=2]}, nodes={auto}]
                (1) -- node{4} (2);
            \draw[-{Stealth[scale=2]}, nodes={auto}]
                (1) -- node{3} (3);
            \draw[-{Stealth[scale=2]}, nodes={auto}]
                (1) -- node{3} (4);
            \draw[-{Stealth[scale=2]}, nodes={auto}]
                (1) -- node{7} (5);
            \draw[-{Stealth[scale=2]}, nodes={auto}]
                (1) -- node{7} (6);
            \draw[-{Stealth[scale=2]}, nodes={auto}]
                (2) -- node{4} (4);
            \draw[-{Stealth[scale=2]}, nodes={auto}]
                (2) -- node{0} (5);
            \draw[-{Stealth[scale=2]}, nodes={auto}]
                (3) -- node{0} (5);
            \draw[-{Stealth[scale=2]}, nodes={auto}]
                (3) -- node{3} (6);
            \draw[-{Stealth[scale=2]}, nodes={auto}]
                (4) -- node{7} (7);
            \draw[-{Stealth[scale=2]}, nodes={auto}]
                (5) -- node{1} (7);
            \draw[-{Stealth[scale=2]}, nodes={auto}]
                (5) -- node{0} (8);
            \draw[-{Stealth[scale=2]}, nodes={auto}]
                (5) -- node{6} (9);
            \draw[-{Stealth[scale=2]}, nodes={auto}]
                (6) -- node{10} (8);
            \draw[-{Stealth[scale=2]}, nodes={auto}]
                (7) -- node{8} (9);
            \draw[-{Stealth[scale=2]}, nodes={auto}]
                (8) -- node{10} (9);
        \end{tikzpicture}
    \end{center}


    \item Построим максимальный поток
    
    \begin{center}
        \begin{tikzpicture}[circ/.style={circle, draw}]
            \path[nodes={circ}]
                (6,  0) node(6) {6}
                (2,  1) node(3) {3}
                (10, 1) node(8) {8}
                (0,  3) node(1) {1}
                (6,  3) node(5) {5}
                (12, 3) node(9) {9}
                (2,  5) node(2) {2}
                (6,  6) node(4) {4}
                (10, 5) node(7) {7}
            ;
            \draw[-{Stealth[scale=2]}, nodes={auto}]
                (1) -- node{4} (2);
            \draw[-{Stealth[scale=2]}, nodes={auto}]
                (1) -- node{3} (3);
            \draw[-{Stealth[scale=2]}, nodes={auto}]
                (1) -- node{3+2+1} (4);
            \draw[-{Stealth[scale=2]}, nodes={auto}]
                (1) -- node{7} (5);
            \draw[-{Stealth[scale=2]}, nodes={auto}]
                (1) -- node{7+3} (6);
            \draw[-{Stealth[scale=2]}, nodes={auto}]
                (2) -- node{4-2-1} (4);
            \draw[-{Stealth[scale=2]}, nodes={auto}]
                (2) -- node{0+2+1} (5);
            \draw[-{Stealth[scale=2]}, nodes={auto}]
                (3) -- node{0+3} (5);
            \draw[-{Stealth[scale=2]}, nodes={auto}]
                (3) -- node{3-3} (6);
            \draw[-{Stealth[scale=2]}, nodes={auto}]
                (4) -- node{7} (7);
            \draw[-{Stealth[scale=2]}, nodes={auto}]
                (5) -- node{1+2} (7);
            \draw[-{Stealth[scale=2]}, nodes={auto}]
                (5) -- node{0+1+3} (8);
            \draw[-{Stealth[scale=2]}, nodes={auto}]
                (5) -- node{6} (9);
            \draw[-{Stealth[scale=2]}, nodes={auto}]
                (6) -- node{10} (8);
            \draw[-{Stealth[scale=2]}, nodes={auto}]
                (7) -- node{8+2} (9);
            \draw[-{Stealth[scale=2]}, nodes={auto}]
                (8) -- node{10+1+3} (9);
        \end{tikzpicture}
    \end{center}

    Найдем увеличивающие цепи:
    \begin{enumerate}
        \item $v_1-v_4-v_2-v_5-v_7-v_9$\\
        $min\{8 - 3, \underline{4}, 3, 4 - 1, 10 - 8\} = 2$
        \item $v_1-v_4-v_2-v_5-v_8-v_9$\\
        $min\{8 - 3 - 2, \underline{4 - 2}, 3 - 2, 4, 15 - 10\} = 1$
        \item $v_1-v_6-v_3-v_5-v_8-v_9$\\
        $min\{10 - 7, \underline{3}, 5, 4 - 1, 15 - 10 - 1\} = 3$
    \end{enumerate}

    Больше увеличивающих цепей нет.
    
\end{enumerate}

\section*{Ответ}

$\Phi_{max} = 10 + 6 + 14 = 30$

\end{document}

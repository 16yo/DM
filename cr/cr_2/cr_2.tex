\documentclass{article}

\usepackage{amsmath}
\usepackage[russian]{babel}

\usepackage{tikz}
\usetikzlibrary{graphs, babel, quotes, calc}

\usepackage[margin=2.5cm]{geometry}


\title{Курсовая работа по дискретной математике\\Вторая задача}
\author{Ахметшин Б.Р. -- М8О-103Б-22 -- 2 вариант}
\date{Март, 2023}

\begin{document}

\maketitle

\section*{Дано}
Граф:

\begin{center}
  \tikz {
    \node [circle, draw] (1) at (2.5,-2.5) {1};
    \node [circle, draw] (2) at (1,  -1.5) {2};
    \node [circle, draw] (3) at (4,     0) {3};
    \node [circle, draw] (4) at (4,  -1.5) {4};
    \node [circle, draw] (5) at (1,     0) {5};
    \graph {
    	(1) -- { (2), (4) };
    	(2) -- { (1), (3), (4), (5) };
    	(3) -- { (2), (4), (5) };
    	(4) -- { (1), (2), (3) };
    	(5) -- { (2), (2) };
    }
  }
\end{center}

\section*{Задание}
Используя алгоритм Терри, определить замкнутый маршрут, проходящий ровно по два раза
(по одному в каждом направлении) через каждое ребро графа

\section*{Решение}
\begin{center}
  \tikz {
    \node [circle, draw, label={150:{*}}] (1) at (2.5,  -3) {1};
    \node [circle, draw, label={ 90:{*}}] (2) at (1,  -1.5) {2};
    \node [circle, draw, label={270:{*}}] (3) at (4,     0) {3};
    \node [circle, draw, label={230:{*}}] (4) at (4,  -1.5) {4};
    \node [circle, draw, label={  0:{*}}] (5) at (1,     0) {5};
    \graph {
    	(1) -- { (2), (4) };
    	(2) -- { (1), (3), (4), (5) };
    	(3) -- { (2), (4), (5) };
    	(4) -- { (1), (2), (3) };
    	(5) -- { (2), (2) };
    };
    \draw[-to] ($(5)!10mm! 15:(3)$) to ($(3)!10mm!-15:(5)$);
    \draw[-to] ($(4)!10mm! -15:(3)$) to ($(3)!10mm!15:(4)$);
    \draw[-to] ($(4)!10mm! 15:(1)$) to ($(1)!10mm!-15:(4)$);
    \draw[-to] ($(1)!10mm! 15:(2)$) to ($(2)!10mm!-15:(1)$);   
    \draw[-to] ($(5)!10mm!-15:(2)$) to ($(2)!10mm! 15:(5)$);   
    \draw[-to] ($(2)!10mm!-15:(5)$) to ($(5)!10mm! 15:(2)$);  
    \draw[-to] ($(2)!10mm!-10:(3)$) to ($(3)!10mm! 10:(2)$);  
    \draw[-to] ($(3)!10mm!-10:(2)$) to ($(2)!10mm! 10:(3)$);  
    \draw[-to] ($(2)!10mm!-10:(4)$) to ($(4)!10mm! 10:(2)$);  
    \draw[-to] ($(4)!10mm!-10:(2)$) to ($(2)!10mm! 10:(4)$); 
    \draw[-to] ($(2)!10mm! 15:(1)$) to ($(1)!10mm!-15:(2)$);
    \draw[-to] ($(1)!10mm! 15:(4)$) to ($(4)!10mm!-15:(1)$);
    \draw[-to] ($(3)!10mm!-15:(4)$) to ($(4)!10mm! 15:(3)$);    
    \draw[-to] ($(3)!10mm! 15:(5)$) to ($(5)!10mm!-15:(3)$);
  }
\end{center}

\section*{Ответ}
В итоге получился такой путь:
$2-1-4-3-5-2-4-2-3-2-5-3-4-1-2$


\end{document}
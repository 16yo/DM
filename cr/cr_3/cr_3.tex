\documentclass{article}

\usepackage{amsmath}
\usepackage[russian]{babel}

\usepackage{tikz}
\usetikzlibrary{graphs, babel, quotes, calc}

\usepackage[margin=2.5cm]{geometry}
\usepackage{graphics}


\title{Курсовая работа по дискретной математике\\Третья задача}
\author{Ахметшин Б.Р. -- М8О-103Б-22 -- 2 вариант}
\date{Март, 2023}

\setcounter{MaxMatrixCols}{20}

\begin{document}

\maketitle

\section*{Дано:}
Матрица смежности орграфа:\\

$
G = 
\begin{pmatrix}
0 && 0 && 0 && 0 && 1 && 1 && 0 \\
1 && 0 && 1 && 0 && 1 && 0 && 1 \\
1 && 1 && 0 && 1 && 1 && 0 && 0 \\
0 && 0 && 1 && 0 && 1 && 1 && 0 \\
1 && 0 && 0 && 1 && 0 && 1 && 0 \\
0 && 0 && 1 && 0 && 1 && 0 && 0 \\
1 && 1 && 0 && 1 && 1 && 0 && 0
\end{pmatrix}
$

\section*{Найти:}

Используя алгоритм “фронта волны”, найти все минимальные пути из первой вершины в последнюю.

\section*{Решение:}

\begin{enumerate}
  \item Помечаем вершину $\nu_1$ индексом 0. Вершина $\nu_1$ принадлежит фронту волны 0 уровня $W_{0}(\nu_1)$.
  \item Вершины из множества $\Gamma_{W_{0}(\nu_1)}=\{\nu_5, \nu_6\}$ помечаем индексом 1, они принадлежат фронту волны 1 уровня $W_{1}(\nu_1)$.
  \item Непомеченные ранее вершины из множества $\Gamma_{W_{1}(\nu_1)}=\Gamma\{\nu_5,\nu_6\}=\{\nu_3,\nu_4\}$ помечаем индексом 2, они принадлежат фронту волны 2 уровня $W_{2}(\nu_1)$.
  \item Непомеченные ранее вершины из множества $\Gamma_{W_{2}(\nu_1)}=\Gamma\{\nu_3,\nu_4\}=\{\nu_2\}$ помечаем индексом 3, они принадлежат фронту волны 3 уровня $W_{3}(\nu_1)$.
  \item Непомеченные ранее вершины из множества $\Gamma_{W_{3}(\nu_1)}=\Gamma\{\nu_2\}=\{\nu_7\}$ помечаем индексом 4, они принадлежат фронту волны 4 уровня $W_{4}(\nu_1)$.
  \item Итак, вершина $\nu_7$ достигнута, помечена индексом 4, следовательно, длина кратчайшего пути из $\nu_1$ в $\nu_7$ равна 4.
\end{enumerate}


\begin{center}
  \includegraphics*[scale=0.4]{./2.png}
\end{center}

Теперь найдем все кратчайшие пути:

\begin{enumerate}
  \item $\nu_7$
  \item $w_{3}(\nu_1) \cap \Gamma^{-1}_{\nu_7}=\{\nu_2\} \cap \{\nu_2\}=\{\nu_2\}$
  \item $w_{2}(\nu_1) \cap \Gamma^{-1}_{\nu_2}=\{\nu_3, \nu_4\} \cap \{\nu_3, \nu_7\}=\{\nu_3\}$
  \item $w_{1}(\nu_1) \cap \Gamma^{-1}_{\nu_3}=\{\nu_5, \nu_6\} \cap \{\nu_2, \nu_4, \nu_6\}=\{\nu_6\}$
  \item $w_{0}(\nu_1) \cap \Gamma^{-1}_{\nu_6}=\{\nu_1\} \cap \{\nu_1, \nu_4, \nu_5\}=\{\nu_1\}$
\end{enumerate}

Кратчайший путь один: $\nu_1 - \nu_6 - \nu_3 - \nu_2 - \nu_7$.

\end{document}
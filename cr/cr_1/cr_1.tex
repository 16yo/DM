\documentclass{article}

\usepackage{amsmath}
\usepackage{enumitem}
\usepackage[russian]{babel}

\title{Курсовая работа по дискретной математике\\Первая задача}
\author{Ахметшин Б.Р. -- М8О-103Б-22 -- 2 вариант}
\date{Март, 2023}

\begin{document}

\maketitle
               
\section*{Дано}
Матрица смежности орграфа
$$
A =
\begin{pmatrix}
  0 && 1 && 1 && 1 \\
  0 && 0 && 0 && 1 \\
  1 && 1 && 0 && 1 \\
  0 && 1 && 0 && 0
\end{pmatrix}
$$

\section*{Найти}
\begin{enumerate}
\item матрицу односторонней связности
\item матрицу сильной связности
\item компоненты сильной связности 
\item матрицу контуров
\end{enumerate}

\section*{Решение}


\subsection*{1.}
\begin{enumerate}		
\item Найдем матрицу односторонней связности при помощи первого алгортима Уоршалла: \\ 
	\[ T = E \lor A \lor ... \lor A^{n-1} \] \\
	\begin{enumerate}
	\item
	$
		E \lor A =
		\begin{pmatrix}
		1 && 1 && 0 && 1 \\ 
		0 && 1 && 0 && 0 \\ 
		0 && 1 && 1 && 1 \\ 
		0 && 0 && 0 && 1 
		\end{pmatrix}
	$
	\item
	$
		A^2 = 
		\begin{pmatrix}
		1 && 1 && 0 && 1 \\ 
		0 && 1 && 0 && 0 \\ 
		0 && 1 && 1 && 1 \\ 
		0 && 0 && 0 && 1
		\end{pmatrix}
		\newline
		E \lor A \lor A^2 = 
		\begin{pmatrix}
		1 && 1 && 1 && 1 \\ 
		0 && 1 && 0 && 1 \\ 
		1 && 1 && 1 && 1 \\ 
		0 && 1 && 0 && 1
		\end{pmatrix}
	$
	\item
	$
		A^3 = 
		\begin{pmatrix}
		0 && 1 && 1 && 1 \\ 
		0 && 0 && 0 && 1 \\ 
		1 && 1 && 0 && 1 \\ 
		0 && 1 && 0 && 0
		\end{pmatrix}
		\newline
		T = E \lor A \lor A^2 \lor A^3 = 
		\begin{pmatrix}
		1 && 1 && 1 && 1 \\ 
		0 && 1 && 0 && 1 \\ 
		1 && 1 && 1 && 1 \\ 
		0 && 1 && 0 && 1
		\end{pmatrix}
	$
	\end{enumerate}
\item Найдем матрицу односторонней связности при помощи итеративного алгортима Уоршалла: \\ 
	\begin{enumerate}
	\item
	$
	  	T^{(0)} = E \lor A =
		\begin{pmatrix}
		1 && 1 && 0 && 1 \\ 
		0 && 1 && 0 && 0 \\ 
		0 && 1 && 1 && 1 \\ 
		0 && 0 && 0 && 1 
		\end{pmatrix}
	$
	\item
	$
		T^{(1)} = ||t^{(1)}_{ij}||, t^{(1)}_{ij} = t^{(0)}_{ij} \lor (t^{(0)}_{i1} \& t^{(0)}_{1j}) =
	  	\begin{pmatrix}
		1 && 1 && 1 && 1 \\ 
		0 && 1 && 0 && 1 \\ 
		1 && 1 && 1 && 1 \\ 
		0 && 1 && 0 && 1
		\end{pmatrix}
	$
	\item
	$
	  	T^{(2)} = ||t^{(2)}_{ij}||, t^{(2)}_{ij} = t^{(1)}_{ij} \lor (t^{(1)}_{i2} \& t^{(1)}_{2j}) =
	  	\begin{pmatrix}
		1 && 1 && 1 && 1 \\ 
		0 && 1 && 0 && 1 \\ 
		1 && 1 && 1 && 1 \\ 
		0 && 1 && 0 && 1
		\end{pmatrix}
	$
	\item
	$
	  	T^{(3)} = ||t^{(3)}_{ij}||, t^{(3)}_{ij} = t^{(2)}_{ij} \lor (t^{(2)}_{i3} \& t^{(2)}_{3j}) =
	  	\begin{pmatrix}
		1 && 1 && 1 && 1 \\ 
		0 && 1 && 0 && 1 \\ 
		1 && 1 && 1 && 1 \\ 
		0 && 1 && 0 && 1
		\end{pmatrix}
	$
	\item
	$
	  	T^{(4)} = ||t^{(4)}_{ij}||, t^{(4)}_{ij} = t^{(3)}_{ij} \lor (t^{(3)}_{i4} \& t^{(3)}_{4j}) =
	  	\begin{pmatrix}
		1 && 1 && 1 && 1 \\ 
		0 && 1 && 0 && 1 \\ 
		1 && 1 && 1 && 1 \\ 
		0 && 1 && 0 && 1
		\end{pmatrix}
	$
	\end{enumerate}
\end{enumerate}
Ответ:
$
T = 
\begin{pmatrix}
1 && 1 && 1 && 1 \\ 
0 && 1 && 0 && 1 \\ 
1 && 1 && 1 && 1 \\ 
0 && 1 && 0 && 1
\end{pmatrix}
$

\subsection*{2.}
$\overline{S} = T \& T^{T} =
\begin{pmatrix}
1 && 1 && 1 && 1 \\ 
0 && 1 && 0 && 1 \\ 
1 && 1 && 1 && 1 \\ 
0 && 1 && 0 && 1
\end{pmatrix}
\&
\begin{pmatrix}
1 && 0 && 1 && 0 \\ 
1 && 1 && 1 && 1 \\ 
1 && 0 && 1 && 0 \\ 
1 && 1 && 1 && 1
\end{pmatrix}
=
\begin{pmatrix}
1 && 0 && 1 && 0 \\
0 && 1 && 0 && 1 \\
1 && 0 && 1 && 0 \\
0 && 1 && 0 && 1
\end{pmatrix}
$
\\
Ответ:
$
\overline{S} =
\begin{pmatrix}
1 && 0 && 1 && 0 \\
0 && 1 && 0 && 1 \\
1 && 0 && 1 && 0 \\
0 && 1 && 0 && 1
\end{pmatrix}
$


\subsection*{3.}
Вершины в первой строке $\overline{S}$ соотвествуют первой компоненте сильной связности,
следовательно первая компонента сильной связности -- $\{v_1, v_3\}$ \Rightarrow
$
\overline{S_1}
=
\begin{pmatrix}
  0 && 0 && 0 && 0 \\
  0 && 1 && 0 && 1 \\
  0 && 0 && 0 && 0 \\
  0 && 1 && 0 && 1
\end{pmatrix}
$
, вторая компонента -- $\{v_2, v_4\}$

Ответ:
$
\{v_1, v_3\}, \{v_2, v_4\}
$

\subsection*{4.}
Матрица контуров вычисляется как:
$
\overline{S} \& A
=
\begin{pmatrix}
  0 && 0 && 1 && 0 \\
  0 && 0 && 0 && 1 \\
  1 && 0 && 0 && 0 \\
  0 && 1 && 0 && 0
\end{pmatrix}
$

Ответ:
$
\begin{pmatrix}
  0 && 0 && 1 && 0 \\
  0 && 0 && 0 && 1 \\
  1 && 0 && 0 && 0 \\
  0 && 1 && 0 && 0
\end{pmatrix}
$

\end{document}